%&pdflatex
\documentclass[10pt,a4paper]
{article}
\usepackage[utf8]{inputenc}
\usepackage{amsmath}
\usepackage{amsfonts}
\usepackage{amssymb}

\setlength{\parskip}{1em}

\begin{document}
\raggedright


\textbf{Cross correlation}

It is the integral of one two time series functions which are offset by some time delay $\tau$
$$R(\tau)= \int_{-\infty}^{+\infty}x(t)y(t+\tau)\delta t $$

If you normalize $R(\tau)$ by the 'energy in the functions' then the maximum value is always 1 (i.e. divide by max value)\par

If the two functions x() and y() are identical then it becomes an autocorrelation. Some basic characteristics of autocorrelation include

\begin{itemize}
  \item white noise only correlates at delay = 0 (no correlation at other lags)
  \item periodic signals taper up to (and away from) a maximum at delay = 0, i.e. as more (and then less) of the signal progressively overlaps with itself.
\end{itemize}

Now, if in the case of cross-correlation (i.e. non-identical functions) then

\begin{itemize}
  \item white noise shows zero correlation at all delays
  \item periodic signals show the same behaviour (but you also get the offset)
\end{itemize}

In real life there will be noise in the data set, which you might want to smooth out (e.g. boxcar average) after applying the cross correlation





\end{document}